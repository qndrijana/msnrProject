\section{Inicijative i zajednice koje se bave rodnom ravnopravnošću u informatici}

Na osnovu podataka koji su prikupljeni o ženama u IT-ju kreirani su efikasni programi koji za cilj imaju povećanje broja žena u ovoj oblasti. 
Univerziteti širom sveta su počeli da menjaju svoje programe vezane za računarstvo sa ciljem da postanu ,,privlačniji'' za žene. Takođe, bilo je bitno dopreti i do devojaka koje su u osnovnim i srednjim školama kako bi im stvorili interesovanje za računarstvo. 

Na primer, Odeljenje za računarske nauke na \emph{Virginia Tech}-u je organizovalo radionice za nastavnike i posete srednjim školama. Odeljenje za računarske nauke na Santa Klara univerzitetu je pokrenulo Letnji inženjerski seminar (eng. \emph{Summer Engineering seminar - SES}) koji nudi srednjoškolcima kamp koji ih upoznaje sa nekoliko polja inženjerstva, uključujuci i računarstvo \cite{hamilton2016gender}. Odeljenje za računarske nauke na Teksaškom univerzitetu u Ostinu je dupliralo broj studentkinja u jednoj godini tako što se fokusiralo na ,,regrutovanje'' devojaka iz srednjih škola. Dodatno, nudili su stipendije brucoškinjama, kao i dobitnicima \emph{NCWIT Aspirations in Computing} nagrade \cite{hamilton2016gender}. 
Stenford je načinio veliki napredak redizajnirajući uvodne kurseve i čineći ih pristupačnijim široj publici i sa 12.5\% žena u studentskoj populaciji 2008, došli su do 21\% 2013. \cite{hamilton2016gender}

Jedan od najvećih uspeha nosi \emph{Harvey Mudd College}. Ovaj koledž je uspeo da poveća broj žena koje studiraju računarstvo sa 12\% na 40\%. U tome su uspeli uz 3 promene: zamenili su tradicionalni CS1 predmet drugačijim pristupom, uveli su letnja istraživanja usmerena na žene druge godine i odvodili su veliki broj studenata prve godine na Grace Hopper proslavu (eng. \emph{Grace Hopper Celebration - GHC}) \cite{hamilton2016gender}.

Ovakve inicijative nisu ograničene samo na koledže: tabelarni prikaz nekih organizacija koje su se bavile ovim pitanjem je dostupan u tabeli \ref{tab:rr}. Jedna od prvih takvih organizacija bila je Udruženje žena u Računarstvu (eng. \emph{Association for Women in Computing - AWC}) koja je osnovana 1978. godine i čija je svrha da pruži prilike za profesionalan rast žena kroz različite programe. Takođe, ova organizacija dodeljuje nagradu \emph{Ada Lovelace} za izvanredno naučno-tehničko dostignuće ili uslugu računarskoj zajednici kroz dostignuća i doprinose u ime žena u računarstvu.
	 
1987. godine, Anita Borg i drugih 12 žena su osnovale \emph{Systers} zajednicu koja je omogućavala ženama da diskutuju o problemima na koje su nailazile na poslu i međusobno dele resurse. Zatim, 1994. godine Anita i dr Tel Vitni osnivaju Grejs Hoper proslavu koja postaje najveće svetsko godišnje okupljanje žena u ovoj oblasti. GHC je osnovan sa ciljem da ženama pruži šansu da poboljšaju svoje veštine, da se upoznaju, međusobno povežu jedna sa drugom, sarađuju i prikažu svoj rad. Naposletku, 1997. godine, Anita osniva neprofitnu organizaciju, originalno poznatu kao Institut za Žene i Tehnologiju (\emph{IWT - Institute for Women and Technology}). Cilj ove organizacije bio je povećanje broja žena na ovom polju, omogućavanje kreiranja većeg broja tehnologija od strane žena i obezbeđivanje da ženski glas ima ulogu u oblikovanju budućnosti tehnologija. Nakon smrti Anite Borg, IWT je preimenovan u Anita Borg institut u njenu čast, a izvršna direktorka postaje Tel. Za vreme njenog upravljanja, institut je rastao i za ovo vreme je GHC postao najuticajnija konferencija za žene tehnologe koja je 2002. brojala svega 630 učesnika, a 2018. preko 22.000 \cite{anitaborg}.

Centar za žene u tehnologiji (eng. \emph{The Center for Women in Technology - CWIT}) je osnovan na Univerzitetu \emph{Maryland Baltimore County - UMBC} u julu 1998. u cilju ostvarivanja punog učešća žena u svim aspektima informacionih tehnologija. Ova organizacija pomaže univerzitetu da postigne svoju misiju tako što identifikuje oblasti u nauci, tehnologiji i inženjerstvu u kom žena nema dovoljno i stipendiranjem dovodi visoko kvalifikovane studentkinje na Univerzitet (\emph{CWIT Scholars Program}) \cite{cwit}.

Nacionalni centar za žene i informacione tehnologije (eng. \emph{National Center for Women \& Information Technology - NCWIT}) je trenutno vodeća organizacija u podršci ulasku i zadržavanju žena u računarstvu. U njihovom istraživanju, došli su do zaključka da je ohrabrenje jedno od ključnih elemenata koje pomaže tome da se žene priključe oblastima u kojima dominiraju muškarci i shodno tome, razvili su program pod nazivom \emph{Aspirations in Computing} \cite{dubow2013bringing}. Ovo je samo jedan od programa razvijenih od strane ove organizacije (\emph{Counselors for Computing}, \emph{BridgeUP STEM} i drugi). Takodje, NCWIT je kreirao \emph{Aspirations Award} koja se dodeljuje na osnovu računarskih i IT sposobnosti, veština predvođenja i akademskog uspeha. Primetili su porast od 54\% među prijavljenim devojkama 2013. u odnosu na prethodnu godinu \cite{dubow2013bringing}. NCWIT okuplja i oprema približno 1.500 organizacija širom zemlje \cite{ncwit}.

Iz Velike Britanije potiče još jedna velika organizacija koja se bavi ženama u IT-ju: specijalna grupa Britanskog računarskog društva (eng. \emph{British Computer Society - BCS}) BCSWomen, koja obezbeđuje prilike za povezivanje žena koje rade u IT-ju širom sveta. Grupu je osnovala dr Su Blek 2001. godine, a inspiraciju je pronašla u 2 događaja kojima je prisustvovala: prvi je bio konferencija na kojoj je ona bila jedina žena, a drugi konferencija \emph{,,Networking the networks''} na kojoj je otkrila čari povezivanja sa drugim ženama u nauci i tehnologiji. U okviru ove grupe, korisnici mogu da prisustvuju raznim događajima koji pokrivaju veliki broj tema o tehnologiji, dobiju šansu da budu uključeni u različite IT događaje, kao i pristupe mentorima i treninzima. Ova grupa danas broji više od 60000 članova u 150 zemalja \cite{bcswomen}.

Inicijativa \emph{EUGAIN - European Network For Gender Balance in Informatics} nastaje 2020. sa ciljem da poboljša rodnu ravnotežu u informatici. Željeni ishodi organizacije se fokusiraju na izradu preporuka i smernica koje će pomoći akademskoj zajednici, zakonodavcima i industriji da se poveća broj žena u ovoj oblasti i ojača njihov položaj.

\begin{table}[h!]
\begin{center}
\begin{tabular}{|p{1.12cm}|p{1.2cm}|p{1.45cm}|p{7.8cm}|} \hline
Naziv& Osnivač& Godina osnivanja& Opis\\ \hline
Girl\newline{}geek dinners & Sara Lamb & 2005. & Održava događaje, najčešće u obliku večera praćenih prezentacijama, sa fokusom na opuštenu atmosferu i druženje. Jedino pravilo: muškarci mogu da prisustvuju samo ako ih neka žena pozove, što obezbeđuje da su većina učesnika žene \cite{girlgeek}. \\ \hline
Girl\newline{}develop it & Vanesa Hurst i Sara Čips & 2010. & Pružaju pristupačne programe za odrasle žene zainteresovane za veb i razvoj softvera. Od 2011. ima preko 100.000 učesnika.\\ \hline
Black girls code & Kimberli Brajant & 2011. & Neprofitna organizacija fokusirana na pružanje edukacije mladim Afroamerikankama.\\ \hline
Girls who code & Rešma Sodžani & 2012. & Koriste različite programe u cilju edukovanja i zadržavanja studenata u ovoj oblasti. Preko 50.000 devojaka je učestvovalo u njihovim programima. \\ \hline
\end{tabular}
\caption{Prikaz nekoliko organizacija koje se bave rodnom ravnopravnošću.}
\label{tab:rr}
\end{center}
\end{table}