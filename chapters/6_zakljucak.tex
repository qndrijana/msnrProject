\section{Zaključak}
\label{sec:zakljucak}
Rodna ravnopravnost podrazumeva da u jednom društvu, a samim tim i u svakoj zajednici i organizaciji, treba da postoji jednaka mogućnost kako za muškarce, tako i za žene i osobe drugih rodnih identiteta i smatra se osnovnim ljudskim pravom. Naučna istraživanja i inicijative su i dalje fokusirani isključivo na odnos muškaraca i žena i videli smo da se položaj žena u IT industriji u poslednjih par godina poboljšava, ali je procenat muškaraca još uvek daleko veći. U Srbiji se broj povećava sa svega dvadeset procenata žena, a osamdeset procenata muškaraca. Sve je veći broj inicijativa poput \emph{Women Know IT} i drugih koje pozivaju mlade žene na diskusije gde ih motivišu da se po ugledu na starije koleginice prijavjuju i krenu putem informatike ukoliko je to ono što ih zanima. 

Primećuje se u svetu i u Srbiji na prestižnim fakultetima informatike i matematike gotovo izjednačen broj studentkinja i studenata. Ono što ima udela u tome jeste što u 2022. gotovo sve firme u Srbiji podstiču žene da se prijavljuju za poslove, podjednako tretiraju oba pola, imaju pogodnosti koje ženi daju sigurnost oko toga da može da razvija svoju karijeru i nesmetano osnuje porodicu, što ranije nije bilo moguće. 