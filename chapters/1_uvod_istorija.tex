\section{Uvod}
\label{sec:uvod}
Proteklih godina pratimo ubrzan razvoj IT industrije, kako u svetu, tako i u Srbiji. Sve veći broj mladih odlučuje se za prestižne fakultete u svetu informatike i matematike. Da li ovaj razvoj prate podjednako žene i muškarci? Da li oba pola imaju jednake mogućnosti, plate, pružene prilike? Potvrdan odgovor na oba pitanja je ključan kako bi se obezbedila rodna ravnopravnost u IT-ju.

Zašto postoji veliki jaz u procentu žena u IT industriji u odnosu na muškarce? Da li se radi na ovome i kojim metodama? Pokazuje se da poslednjih godina broj žena jeste u porastu. Naime, sve veći broj studentkinja se opredeljuje za smerove informatike, iako se na osnovu statistika primećuje da još uvek ne postoji ravnopravnost.

\section{Rodna ravnopravnost kroz istoriju}
\label{sec:istorija}
Iako se danas na osnovu dostupne statistike može zaključiti da je informatika prvenstveno ,,muška'' oblast, istorijski gledano, situacija je bila drugačija. Same početke informatike i računarskih nauka obeležile su žene. Mnoge od njih nepravedno su zaboravljene, te se zato danas stiče utisak da se odnos polova u informatici nije menjao kroz vreme.

Ada King Lavlejs (1815-1852), ćerka čuvenog engleskog pesnika Džordža Gordona Bajrona, smatra se prvom osobom koja se bavila programiranjem. Već sa 17 godina upoznala je Čarlsa Bebidža, zajedno sa kojim je radila na nacrtu za diferencijalnu mašinu, kao i za analitičku mašinu za koju je napisala prvi algoritam \cite{brockwell}. Pisala je o Bebidžovom radu gde je predlagala svoje ideje i unapređenja, ističući da ,,računar može da uradi bilo šta što može biti logički uočeno''. Nažalost, preminula je sa samo 36 godina života, te nikad nisu konstruisali mašinu \cite{davis}. Pored toga, ostavila je ogroman trag u razvoju informatike i postala poznata i kao ,,majka programiranja''. Njeni i Bebidževi zapisi korišćeni su za konstruisanje prvog računara čak oko 100 godina kasnije \cite{npr, fuegi}.

Nakon Američkog građanskog rata, žene su zapošljavane kao ,,ljudski računari'', i to najčešće udovice, dok su ostale dobijale priliku kada je falilo muške radne snage. Bile su manje plaćene nego muškarci, dok su neke čak i volontirale, iako su bile jednako sposobne \cite{evans}. Dodatno, većina poslova koje su žene obavljale, školovani muškarci smatrali su nezanimljivim ili nedovoljno plaćenim. Vremenom su žene počele i da se školuju za računarske poslove, te je njihov broj u ovoj oblasti neprestano rastao \cite{sobel}.

{\sloppy
Tokom Prvog svetskog rata, s obzirom na veliki broj mu\-ška\-raca koji su bili na bojnom polju, većina računarskih poslova pripala je ženama. U Velikoj Britaniji vršile su složena izračunavanja, koja su pomogla Britaniji tokom rata. Nakon toga su samo neke uspevale da opstanu kao profesori u srednjim školama, dok je većina ostala bez posla \cite{grier}. Muški školovani inženjeri gledali su na ta neophodna izračunavanja kao na poslove nedostojne njihove ekspertize, jer je ovaj posao smatran činovničkim. Taj trend nastavio se i tokom i nakon Drugog svetskog rata.}

Pored činjenice da je na ENIAC-u, jednom od prvih elektronskih raču\-nara, radilo 6 žena (,,ENIAC žene''), koje se danas proslavljaju kao prve programerke u modernom smislu reči, tada njihov uticaj nije cenjen koliko je zasluženo. Muški inženjeri mahom su bili koncentrisani na hardver, te su razvoj softvera smatrali manje prestižnom oblašću \cite{frink}. Koliko je samo programiranje softvera i uticaj žena bio zanemaren, pokazuje činjenica da su tadašnji mediji u vreme predstavljanja ENIAC-a potpuno zanemarili imena žena koje su učestvovale u projektu \cite{npr}.

Sličan tim programerki je zatim radio na UNIVAC-u (jednom od prvih komercijalnih računara) zajedno sa Grejs Hoper, čuvenom profesorkom matematike i jednom od pionira modernog programiranja. Od nje je potekla ideja da se računarski programi pišu engleskim jezikom, a ne pomoću brojeva, kako su se do tada pisali. Tako je nastao prvi kompajler. Na taj način je uticala da se softver po svojoj značajnosti izjednači sa hardverom, ako ne i da postane bitniji. Sa engleskog jezika program je mogao da se prevodi za bilo koji željeni hardver \cite{npr}. Međutim, iako se status hardvera poboljšao, ovo nije pozitivno uticalo na broj žena u informatici.

Kako je značajnost programiranja postajala sve jasnija, poslodavci su počeli da traže mlade muškarce zainteresovane za računare. Protiv žena su se povremeno vodile i kampanje sa ciljem da se programiranje promoviše kao dominantno muška oblast. ,,Šta ima 16 nogu, 8 brzih jezika i košta te barem 40 000 dolara godišnje?'' bio je samo jedan od naslova. Čak ni Kosmopolitanov članak ,,Devojke računarstva'' nije pomogao da se ovakav trend spreči. Odjednom je programiranje postajalo prestižna i cenjena oblast. Zato su krajem šezdesetih godina prošlog veka nastupile velike promene u odnosu polova u programiranju. Broj žena je i dalje bio značajan, ali su bile znatno manje plaćene \cite{evans, frink}.

{\sloppy
Tokom narednih godina, polako se uspostavljala dominacija broja mu\-ška\-raca u ovom odnosu. Prema nekim autorima, testovi koje su dobijali kandidati za posao bili su namerno više prilagođeni muškarcima \cite{frink}, kreirane su video igre pune akcije tako da su primarno ciljale muškarce koji su činili većinu korisnika računara. Broj žena u računarstvu dostigao je vrhunac sredinom 1984. godine kada je procenat ženskih diplomaca bio 37.1\%. Zatim je usledio konstantan pad, gde je već 1998. taj broj bio 26.7\% \cite{camp}.}

U 21. veku situacija se nije mnogo menjala, te je od početka veka procenat ženskih diplomaca konstatno ispod 20\%. Pokrenut je veliki broj inicijativa i uložena ogromna količina novca sa ciljem da se među devojčicama i adolescentkinjama promovišu računarske nauke, sa ciljem da se jaz među polovima umanji \cite{aei}. Takođe je zastupljeniji naučni rad u ovoj oblasti, sa više konferencija i časopisa koji su posvećeni isključivo problemima rodne ravnopravnosti u informatici i uopšte u MINT disciplinama (matematika, inženjerstvo, nauka i tehnologija, od eng. \emph{STEM}) \cite{soe-yakura}.

Veliki broj istraživača ovog odnosa kroz istoriju smatrao je da su relevantni uzori jedan od glavnih faktora koji utiče na to da li će se mlade devojke odlučiti za programiranje. S obzirom da istorija nepravedno zanemaruje trag koji su žene ostavile u programiranju, posebno u samim počecima, broj istorijskih uzora nije veliki. Međutim, postavlja se pitanje da li je to zaista presudno, i koji su sve faktori koji utiču na tako veliki jaz među polovima? U svakom slučaju, ako se na osnovu istorije nešto može zaključiti, to je da nema opravdanja da se informatika smatra profesijom predodređenom za bilo koji od polova, bez obzira na statističke podatke.