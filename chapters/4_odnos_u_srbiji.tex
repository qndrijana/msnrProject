\section{Rodna ravnopravnost u informatici u Srbiji}
\label{sec:usrbiji}
Od 1489 prijavljenih za program ,,Prekvalifikacije za IT'' u organizaciji kabineta predsednice Vlade Srbije, 55\% su žene. U okviru programa ,,Škole za 21. vek'', British Council je razvio vodič i niz aktivnosti kako bi pomogao osnovnim školama da osnuju sekcije za programiranje kao vannastavnu aktivnost. Ovaj program pohađaju podjednako devojčice i dečaci, što je indikator da iako su istorijski žene u Srbiji činile svega 20 do 30\% IT industrije, broj je u porastu \cite{veb1}. Ukoliko se osvrnemo na podatke iz naše okoline, vidimo da je sve veći broj studentkinja koje se prijavljuju za tehničke i prirodno-matematičke fakultete, ali su devojke i dalje u manjini. 

Majkrosoft Razvojni Centar u Srbiji je pokrenuo zanimljivu inicijativu pod nazivom \emph{Women know IT}. Grupa već uspešnih žena iz IT industrije, bilo da su to softverske inženjerke ili žene iz \emph{HR}-a, odlučile su da naprave pokret koji će pozivati sve više mladih devojaka da vide pogodnosti ovog posla, da se ohrabre da krenu putem IT-a iako trenutno deluje da je to muški posao zbog ogromne razlike u procentu muškaraca u odnosu na žene u ovoj industriji. Pokrenute su pre svega jednodnevne radionice u kojima se provodi dan sa studentkinjama ili devojkama koje su u srednjim školama. U toku tog dana, postoje diskusije o poslu, pogodnostima, mogućnostima napredovanja. Mlade devojke uče da sve više žena postaju menadžerke i jako brzo napreduju i utiču na razvoj firme \cite{veb3}.

Pored ovoga, problematika koju primećujemo jeste što iako u Srbiji postoji sve više IT stručnjaka koji rade u stranim startapima, te rade od kuće uz pogodnosti visokih plata, jako je čest slučaj da ove strane kompanije nemaju sedište u Srbiji i ne pružaju mogućnost plaćenog porodiljskog odustva. Zbog toga ima više slučajeva žena u našoj okolini koje se ili isključivo okreću domaćim firmama ili napuštaju svoje karijere u usponu da bi se posvetile deci, a kasnije nije lako krenuti ispočetka u novoj firmi.

Još jedan problem o kom se priča jeste što postoje nezvanične informacije da u nekim firmama u Srbiji, za iste pozicije i završene fakultete žene su bile plaćane manje od muškaraca. Ono što devojčice od malena mogu da primete jeste da nema puno ženskih velikih uzora u IT industriji, čime nisu ohrabrene da krenu tim putem, pored svih ovih priča i verovanja da će biti manje plaćene, kako u stranim državama tako i u Srbiji.

Situacija se popravlja a rodna razlika se smanjuje. Međutim, ako zaista nameravamo da zadovoljimo potrebe ekonomije 21. veka, moramo srušiti barijere, povećati interesovanje devojaka za MINT i podstaći više žena da pokrenu ili nastave karijeru u tom pravcu \cite{veb2}. U nastavku ćemo prikazati neke od načina na koji se ovaj problem rešava u Evropi i Americi i koje možemo primeniti i u Srbiji, ali kvalitetni i konkretni predlozi zahtevaju više istraživanja o tome šta bi moglo da funkcioniše na ovim prostorima.