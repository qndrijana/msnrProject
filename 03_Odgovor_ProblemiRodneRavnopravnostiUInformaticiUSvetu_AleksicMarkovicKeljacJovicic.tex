 

 % !TEX encoding = UTF-8 Unicode

\documentclass[a4paper]{report}

\usepackage[T2A]{fontenc} % enable Cyrillic fonts
\usepackage[utf8x,utf8]{inputenc} % make weird characters work
\usepackage[serbian]{babel}
%\usepackage[english,serbianc]{babel}
\usepackage{amssymb}

\usepackage{color}
\usepackage{url}
\usepackage[unicode]{hyperref}
\hypersetup{colorlinks,citecolor=green,filecolor=green,linkcolor=blue,urlcolor=blue}

\newcommand{\odgovor}[1]{\textcolor{blue}{#1}}

\begin{document}

\title{Problemi rodne ravnopravnosti u informatici u svetu\\ \small{Andrijana Aleksić, Bogdan Marković, Jelena Keljać, Luka Jovičić}}

\maketitle

\tableofcontents

% \chapter{Uputstva}
% \emph{Prilikom predavanja odgovora na recenziju, obrišite ovo poglavlje.}

% Neophodno je odgovoriti na sve zamerke koje su navedene u okviru recenzija. Svaki odgovor pišete u okviru okruženja \verb"\odgovor", \odgovor{kako bi vaši odgovori bili lakše uočljivi.} 
% \begin{enumerate}

% \item Odgovor treba da sadrži na koji način ste izmenili rad da bi adresirali problem koji je recenzent naveo. Na primer, to može biti neka dodata rečenica ili dodat pasus. Ukoliko je u pitanju kraći tekst onda ga možete navesti direktno u ovom dokumentu, ukoliko je u pitanju duži tekst, onda navedete samo na kojoj strani i gde tačno se taj novi tekst nalazi. Ukoliko je izmenjeno ime nekog poglavlja, navedite na koji način je izmenjeno, i slično, u zavisnosti od izmena koje ste napravili. 

% \item Ukoliko ništa niste izmenili povodom neke zamerke, detaljno obrazložite zašto zahtev recenzenta nije uvažen.

% \item Ukoliko ste napravili i neke izmene koje recenzenti nisu tražili, njih navedite u poslednjem poglavlju tj u poglavlju Dodatne izmene.
% \end{enumerate}

% Za svakog recenzenta dodajte ocenu od 1 do 5 koja označava koliko vam je recenzija bila korisna, odnosno koliko vam je pomogla da unapredite rad. Ocena 1 označava da vam recenzija nije bila korisna, ocena 5 označava da vam je recenzija bila veoma korisna. 

% NAPOMENA: Recenzije ce biti ocenjene nezavisno od vaših ocena. Na osnovu recenzije ja znam da li je ona korisna ili ne, pa na taj način vama idu negativni poeni ukoliko kažete da je korisno nešto što nije korisno. Vašim kolegama šteti da kažete da im je recenzija korisna jer će misliti da su je dobro uradili, iako to zapravo nisu. Isto važi i na drugu stranu, tj nemojte reći da nije korisno ono što jeste korisno. Prema tome, trudite se da budete objektivni. 
\chapter{Recenzent \odgovor{--- ocena: 5} }


\section{O čemu rad govori?}
% Напишете један кратак пасус у којим ћете својим речима препричати суштину рада (и тиме показати да сте рад пажљиво прочитали и разумели). Обим од 200 до 400 карактера.
U ovom radu je diskutovano na temu rodne ravnopravnosti u oblasti informatike, ali i srodnim naukama. Objašnjeni su osnovni uzroci, koji kroz istoriju, dovode do ovog problema. Upoređene su statistike iz različitih delova sveta, kao i koraci koji su preduzeti kako bi se ovaj problem rešio. Primećen je značajan napredak poslednjih godina u nasoj zemlji i inostranstvu zahvaljujući svim tim preduzetim merama.

\section{Krupne primedbe i sugestije}
% Напишете своја запажања и конструктивне идеје шта у раду недостаје и шта би требало да се промени-измени-дода-одузме да би рад био квалитетнији.
Prvenstveno, smatram da je rad veoma dobro odgovorio na temu, obradio je sve bitne probleme, od samog početka razvoja ovih oblasti, pa do danas, u različitim delovima sveta. Iz načina na koji je rad napisan vidi se da su kolege pronašle dosta korisnih informacija i organizovale na takav način da je rad razumljiv i lak za čitanje. Lepo su prikazali kako se postepeno, primenom različitih mera (osnivanja organizacija, konferencija, kao i drugih aktivnosti kojim se podiže svest o ovom problemu), situacija poboljšava iz godine u godinu, međutim, takođe je primećeno da 
statistike i dalje nisu zadovoljavajuće, da su muškarci i dalje brojniji.
Dosta zanimljivih činjenica se može saznati iz priče o istorijatu, ali i aktivnostima koje se preduzimaju kako bi se ovi problemi rešili. Rad sadrži dosta statističkih podataka, ali to ga ne čini napornim za čitanje jer se na dobar način ilustruje situacija o kojoj se priča, a takođe sve pomenute sistematizacije su ukratko pojašnjene.

Kada su u pitanju zamerke: rad ima krupnijih stilskih gresaka. Odaje utisak da autori nisu pročitali rad u celini kako bi usaglasili način na koji pisu, već se primećuju stilske razlike. Može negde primetiti prisustvo ličnog stava samih autora ovog rada, mada bi bilo bolje da toga ima više. 

\odgovor{Jasan nam je sentiment komentara koji se tiče stilskih razlika, međutim nismo sigurni koje konkretne ispravke bi doprinele njihovom smanjivanju i sveukupno konzistentnije napisanom radu. Ispravili smo način citiranja u celom radu tako da se reference nalaze na kraju delova na koje se odnose (umesto kod imena radova) i prebacili rečenice koje name potrebe da budu u pasivu u aktiv. Takođe smo se tokom odgovaranja na druge primedbe (u ovoj i drugoj recenziji) trudili da budemo ujednačeniji po stilskim pitanjima. S druge strane, formulacija pojedinih rečenica često nije bilo isključivo stilsko pitanje: recimo, kada citiramo radove iz socijalne psihologije, bili smo pažljivi da rezultate jednog rada ne predstavimo kao nepobitne činjenice (te su rečenice oblika ,,[autori] tvrde/misle/su pronašli da [nešto]''), dok su istorijski podaci često pouzdaniji i mislimo da nema potrebe da čitaoca opterećujemo takvim detaljima i formulacijama. Slično, za tekst o organizacijama i inicijativama smo se najčešće oslanjali na podatke koje oni sami objavljuju, tako da oni često nešto nastoje i pokušavaju, a njihovi ciljevi su drugačijeg formata od ciljeva naučnih istraživanja.
Odsustvo ličnog stava je svestan izbor --- kako smo mi kao autori tek na počecima karijera, smatramo da je naš doprinos vredniji ako se fokusira na pregled literature i povezivanje dostupnih (objektivnih, empirijskih) činjenica, kao i njihova selekcija i predstavljanje u formatu koji je prijemčiv čitaocu. Izražavanje ličnih mišljenja i iskustava, postavljanje hipoteza i teorija prepuštamo istraživačima sa više decenija relevantnog rada u polju, a svoje preferiramo da iskazujemo u ograničenom obimu u ovom formatu.}

\section{Sitne primedbe}
% Напишете своја запажања на тему штампарских-стилских-језичких грешки
\begin{enumerate}
\item
strana 1, “U ovom radu nudimo istorijski osvrt na pitanje rodne ravnopravnosti u informatici i opis trenutnog stanja u ovoj oblasti:“ \\
Predlažem izmenu: “U ovom radu prikazujemo istorijski osvrt na temu rodne ravnopravnosti u informatici i opis trenutnog stanja u ovoj oblasti:“

\odgovor{Izmenjeno po predlogu.}
\item
strana 1, “Izlažemo pregled literature koja opisuje i pokušava da objasni ovaj problem, počev od obrazovnog sistema pa do poslodavaca u industriji ili akademiji.“ \\
Predlažem izmenu: “Izlažemo pregled literature koja objašnjava ovaj problem, počev od obrazovnog sistema, pa do poslodavaca u industriji ili akademiji.“

\odgovor{Izmenjeno po predlogu.}
\item
strana 1: “metode i rezultate koji nas dovode do ravnopravnije
zajednice, a i društva, od čega će korist naposletku imati svi.“ \\
Predlažem izmenu: “metode i rezultate koji nas dovode do ravnopravnijeg
društva, što će koristiti svima.“

\odgovor{Izmenjeno po predlogu.}
\item
strana 2: “Zašto je veliki jaz“  \\
Predlažem izmenu: “Zašto postoji veliki jaz“

\odgovor{Izmenjeno po predlogu.}
\item
strana 2: “Pokazuje se da poslednjih godina
broj žena jeste u porastu, veći broj studentkinja se opredeljuje za
smerove Informatike, mada statistike još uvek pokazuju nedovoljno da
postoji ravnopravnost.“  \\
Predlažem izmenu: “Pokazuje se da poslednjih godina
broj žena jeste u porastu. Naime, sve veći broj studentkinja se opredeljuje za smerove informatike, iako se na osnovu statistika primecuje da jos uvek ne postoji ravnopravnost.“

\odgovor{Izmenjeno po predlogu.}
\item
strana 2, “istorijski ovo nije bilo tako“  \\
Predlažem izmenu: “istorijski gledano, situacija je bila drugacija“

\odgovor{Izmenjeno po predlogu.}
\item
strana 3, “procenat ženskih diplomaca konstatno ispod 20 posto“  \\
Predlažem izmenu: “procenat ženskih diplomaca konstatno ispod 20\%“ 

\odgovor{Izmenjeno po predlogu.}
\item
strana 3, reč “muškaraca“ izlazi iz okvira desne margine dokumenta

\odgovor{\LaTeX{} odbija da prelomi ovu reč (čak i uz \texttt{\textbackslash{}-} posle slogova). Dodali smo \texttt{\textbackslash{}sloppy} koji se odnosi samo na ovaj pasus.}
\item
strana 3, “broj lako uočivih istorijskih uzora nije veliki“  \\
Predlažem izmenu: “broj poznatih istorijskih uzora nije veliki“

\odgovor{Izmenjeno po predlogu.}
\item
strana 3, “jaz među polovima.“  \\
Predlažem izmenu: “jaz među polovima?“

\odgovor{Izmenjeno po predlogu.}
\item
strana 4, “Pošto se u se u literaturi često ne postoje precizniji termini, koristićemo grupe srodnih oblasti u prezentaciji podataka i zaključaka kada ne postoji preciznija alternativa.“  \\
Predlažem izmenu: “Pošto u literaturi često ne postoje precizniji termini, koristićemo srodne oblasti u prezentaciji podataka i zaključaka.“

\odgovor{Izmenjeno po predlogu.}
\item
strana 4, “umesto traženja rešenja za jedno ili nekoliko faza“ \\
Predlažem izmenu: “umesto traženja rešenja za konkretne probleme“

\odgovor{Želeli smo da naglasimo zašto autorke smatraju da analogija cevi, tj. rešavanje problema po ,,fazama'' kroz koje se prolazi, nije adekvatno. Međutim, vidimo zašto rečenica sročena na ovaj način može biti konfuzna, tako da smo je izmenili i sada stoji: ,,...za probleme u jednoj ili nekoliko faza u obrazovanju ili zapošljavanju (npr. fokusiranje samo na fakultet ili zapošljavanje)...''}
\item
strana 4, “kritičnu masu žena, ili muškaraca“  \\
Predlažem izmenu: “određeni broj žena i muškaraca“

\odgovor{Zamenili smo zapetu i veznik sa ,,i/ili'', što čini rečenicu malo jednostavnijom za čitanje i smatramo da i dalje odgovara ideji iz citiranog rada. ,,Kritična masa'' je termin iz citiranog rada, koji iako zvuči neodređeno, smatramo da bolje opisuje koncept nego ,,određeni''.}
\item
strana 5,  inžinjerskim - inženjerskim

\odgovor{Ispravljeno.}
\item
strana 5, “Takođe su učestale reference na maskulinu štrebersku kulturu i propagiranje rodnih stereotipa.“  \\
Ne razumem smisao ove rečenice.

\odgovor{Rečenica zamenjena sa: ,,Takođe su učestale reference na štrebersku (eng. \emph{geek}) kulturu, koja je po prirodi maskulina, kao i propagiranje rodnih stereotipa, što ima veliki potencijal da odbije ženske kandidate u startu.''}
\item
strana 6, “On predlaže sličan pristup i za druge države ponaosob, pošto
smatra da se pol ne može posmatrati u izolaciji i da isti pristup ne mora
nužno da funkcioniče svuda.“   \\
Ne razumem ovu rečenicu.

\odgovor{Rečenica je izbrisana kao rezultat odgovora drugom recenzentu.}
\item
strana 6, “verodostojne cifre za prosek“  \\  
Predlažem izmenu: “verodostojne vrednosti proseka“

\odgovor{Izmenjeno po predlogu}
\item
strana 6, Sekcije - sekcije

\odgovor{Ispravljeno}
\item
strana 7, “Mlade devojke uče da su sve više žene i menadžeri i jako brzo napreduju i utiču na razvoj firme.“ \\   
Ne razumem ovu rečenicu

\odgovor{Zamenjeno sa ,,Mlade devojke uče da sve više žena postaju menadžerke i jako brzo napreduju i utiču na razvoj firme.''}
\item
strana 7, “Stvari se menjaju nabolje, situacija se popravlja“ \\  
Predlažem izmenu: dovoljo je reći samo situacija se popravlja“

\odgovor{Izmenjeno po predlogu.}
\item
strana 7, Poslednja dva pasusa koriste se na dva mesta jednostruki navodnici, prebaciti u dvostruke

\odgovor{Ispravljeno.}
\end{enumerate}


\section{Provera sadržajnosti i forme seminarskog rada}
% Oдговорите на следећа питања --- уз сваки одговор дати и образложење

\begin{enumerate}
\item Da li rad dobro odgovara na zadatu temu?\\
Da, diskutovano je o svim važnim pitanjima, preduzetim merama i efikasnosti tih mera, a zaključak lepo zaokruzuje temu.
\item Da li je nešto važno propušteno?\\
Diskutovano je o svim suštinski bitnim pitanjima, mozda je trebalo još malo više da se kaže o EUGAIN inicijativi, ali i bez toga je potpun.
\item Da li ima suštinskih grešaka i propusta?\\
Smatram da nema suštinskih propusta u sadržaju rada.
\item Da li je naslov rada dobro izabran?\\
Naslov glasti isto kao i tema na koju je pisan rad, međutim ovo je svakako interesantna tema, tako da neki kreatiniji naslov nije od presudnog značaja u cilju privlačenja citalaca. 
\item Da li sažetak sadrži prave podatke o radu?\\
Sadrži ukratko opisanu glavnu ideju ove teme i time čitaoca uvodi u temu i priprema za dalju diskusiju.
\item Da li je rad lak-težak za čitanje?\\
Rad je lak za čitanje, teme se lepo nadovezuju, međutim brojne stilske greške čine rad konfuznim u nekim delovima.
\item Da li je za razumevanje teksta potrebno predznanje i u kolikoj meri?\\
Smatram da svako može da razume ovaj rad bez predznanja.
\item Da li je u radu navedena odgovarajuća literatura?\\
Da, sve tvrdnje su lepo argumentovane korišćenjem adekvatne literature.
\item Da li su u radu reference korektno navedene?\\
Reference koje su navođene na kraju pasusa ili nekoliko rečenica su navođene nakon kraja istih, posle tačke, a trebalo bi pre. Na mestima gde se navodi više referenci u radu je navođeno [?][?] umesto [?,?].

\odgovor{Ispravljeno.}
\item Da li je struktura rada adekvatna?\\
Struktura rada jeste adekvatna i jedina zamerka je što na tri mesta postoje paragrafi koji sadrže samo po jednu, izuzetnu dugačku, rečenicu (poglavlje 3.1 prvi pasus, poglavlje 4 poslednji pasus, poglavlje 6 drugi pasus).

\odgovor{U poglavlju 3.1 smo spojili prva dva pasusa. Rečenicu u poslednjem pasusu u poglavlju 4 smo razdvojili na dve, zamenivši ,,, ali'' sa ,,. Međutim, ''. U poglavlju 6 smo prebacili poslednju rečenicu iz prvog pasusa na početak drugog pasusa.}
\item Da li rad sadrži sve elemente propisane uslovom seminarskog rada (slike, tabele, broj strana...)?\\
Da, postoji jedna tabela, jedna slika i više nego dovoljno referenci.
\item Da li su slike i tabele funkcionalne i adekvatne?\\
Odabrana je pogodna statistika da se napravi grafik (slika), međutim nije neophodno da postoje labele koje imenuju ose (Procenat, Pozicija), već se to moze napisati samo u okviru opisa ispod slike, kako bi slika izgledala lepse i preglednije. Labele na x osi slike bi trebalo da sve budu u liniji (a ne ovako cik-cak). Tabela takođe prikazuje lepo odabrane podatke, ali bi trebalo da izgleda preglednije (bez prelamanja reči) i opis tabele bi lepše izgledao ispod tabele.

\odgovor{Izbačeni su nazivi osa na slici i prebačeni u opis. Labele na x osi idu u cik-cak zbog ograničenog prostora i načina na koji \texttt{ggplot}, pomoću kojeg je slika pravljena, generiše grafikon. Pomerili smo opis tabele tako da se nalazi ispod nje i promenili smo veličine kolona tako da izbegnemo prelamanje, a gde to nije moguće, u uske kolone smo ručno dodali nove redove (prelamanja u koloni ,,Opis'' i dalje postoje i ne mislimo da bi njihovo eliminisanje povećalo čitljivost).}
\end{enumerate}

\section{Ocenite sebe}
% Napišite koliko ste upućeni u oblast koju recenzirate: 
% a) ekspert u datoj oblasti
% b) veoma upućeni u oblast
% c) srednje upućeni
% d) malo upućeni 
% e) skoro neupućeni
% f) potpuno neupućeni
% Obrazložite svoju odluku
Srednje upućena - pročitala sam ranije nekoliko interesantnih članaka na ovu temu, diskutovala sa kolegama/prijateljima nekoliko puta i sada se upoznala sa nekim stvarima za koje nisam znala ranije.

\chapter{Recenzent \odgovor{--- ocena: 5} }


\section{O čemu rad govori?}
% Напишете један кратак пасус у којим ћете својим речима препричати суштину рада (и тиме показати да сте рад пажљиво прочитали и разумели). Обим од 200 до 400 карактера.
Seminarski rad nam govori o problemima rodne ravnopravnosti u informatici u svetu ali i u Srbiji. Poslednjih godina, sva društva i države u svetu bave se temom rodne ravnopravnosti jer se pokazalo da je to kroz istoriju bila tema koja se zanemarivala. Ideja bavljenja ovom temom je da se na odgovarajući način pokaže poštovanje ženama koje se bave najrazličitijim profesijama, posebno informatikom.

\section{Krupne primedbe i sugestije}
% Напишете своја запажања и конструктивне идеје шта у раду недостаје и шта би требало да се промени-измени-дода-одузме да би рад био квалитетнији.
Što se tiče većih primedbi koje imam na celokupan seminarski, pokušaću da kroz teme iz sadržaja iznesem svoje primedbe i dam sugestije.

U \textbf{uvodu} je trebalo napisati opštu definiciju rodne ravnopravnost kako bi i onaj ko uopšte nije upoznat šta to znači i gde je ima ili nema, mogao da sa razumevanjem pročita i razume rad.
\
odgovor{Dodali smo rečenicu ,,Potvrdan odgovor na oba pitanja je ključan kako bi se obezbedila rodna ravnopravnost u IT-ju.'' na kraj prvog pasusa, kao način da definišemo ovaj pojam preko problema na koji se fokusira. Nismo sigurni kako bolje da definišemo ovaj koncept bez ulaženja u semantiku reči ,,rod'' i ,,ravnopravnost'', što smatramo suvišnim i potencijalno snishodljivim prema čitaocu. U literaturi nismo primetili da se ovaj koncept često posebno direktno definiše.}

U delu rada koji govori o \textbf{rodnoj ravnopravnosti kroz istoriju} je trebalo napomenuti da je to tema o kojoj se tek poslednjih desetak godina posvećuje više pažnje, i teoretski i praktično. Takođe, trebalo je napisati nešto više o ženama koje su doprinele inovacijama u informatici kako bi se videlo da li su one manje cenjene od muškaraca inovatora. Zna se da su informatičarkama plate niže za isti posao od kolega već decenijama, trebalo je pokušati pronaći više podataka o razlozima za tako nešto.

\odgovor{Dodali smo rečenicu ,,Takođe je zastupljeniji naučni rad u ovoj oblasti, sa više konferencija i časopisa koji su posvećeni isključivo problemima rodne ravnopravnosti u informatici i uopšte u MINT disciplinama (matematika, inženjerstvo, nauka i tehnologija, od eng. \emph{STEM})'' na kraj pretposlednjeg pasusa u ovom delu, a izbacili smo sličnu rečenicu iz prvog pasusa u trećem odeljku. Slažemo se da bi prikaz više dobrih primera doprineo radu, ali s obzirom na širu temu rada i ograničen prostor, ne smatramo da je opravdano izbaciti neki drugi deo nauštrb toga. U narednom odeljku smo dali razloge zašto su žene diskriminisan i manje povlašćen pol i to direktno povezali sa platama; konkretni razlozi koji bi se odnosili na plate bi zahtevali detaljniju ekonomsku analizu koja će se razlikovati od sistema do sistema (za neregulisanija tržišta poput privatnog sektora u SAD-u, teško je opovrgnuti razloge koji su varijacija na temu ,,zato što je poslodavac tako odredio, jer smatra da to odgovara biznisu i/ili njemu/njoj lično'', dok za striktno regulisana tržišta gde je plata fiksna i poznata unapred za svaku poziciju nema smisla postavljati ova pitanja). Pitanje, iako veoma zanimljivo, verovatno zahteva jedan rad ove dužine kako bi se adekvatno adresiralo, a ne želimo da rizikujemo da ga predstavimo na pogrešan način.}

\textbf{Odnos polova u informatici} dat je kao rezultat istraživanja u IT kompanijama, preduzećima i univerzitetima koji su bili dostupni na internetu i to uglavnom u Americi a veoma malo u Evropi i Srbiji. Nije poznat način odnosno metod prikupljanja podataka, da li je istraživanje sprovedeno anonimnim anketiranjem ili na drugi način, IT profesionalaca zaposlenih u IT kompanijama i univerzitetima. Rodna neravnovopravnost se može sagledati i poređenjem poslova u informatici u IT kompanijama, trebalo je naći primer na kojim poslovima i radnim zadacima rade većinom žene a koje poslove pokrivaju muškarci.

\odgovor{Dodali smo ,,i to prvenstveno na pozicijama projektne menadžerke i biznis analitičarke (dok su muškarci češće softverski inžinjeri i sistem administratori)'' na kraj pretposlednje rečenice u prvom pasusu (podaci su iz citiranog rada) kako bismo čitaocu dali nešto detaljniji uvid. Za Evropu ovako preciznu podelu nismo uspeli da nađemo na bilo kojoj značajnijoj skali, a navođenje konkretnih cifara bez dovođenja čitaoca u zabludu je generalno problem, s obzirom da nazivi pozicija nisu standardizovani u industriji. Ova tema je mnogo više zastupljena u američkoj literaturi, ali smo se trudili da statističke podatke iznosimo za oba kontinenta. Za teorijske uvide, jedinu značajnu razliku vidimo u grupi disciplina sa kojima se informatika predstavlja (MINT/IKT), ali nismo ovu razliku smatrali značajnom i birali smo one izvore koji nam daju bolji uvid (koji su pretežno američki). S obzirom na temu rada (a i ovog odeljka), Srbija nam nije bila u fokusu. Zbirni podaci za celu Evropu su najčešće ograničeni samo na Eurostat; nismo našli veća internacionalna istraživanja. Nismo posebno naglašavali metodologiju zvaničnih podataka, već podatke državnih i međudržavnih agencija smatramo verodostojnim jer su po pravilu rađeni na celokupnoj populaciji, a ne na uzorku; u slučaju podataka nevladinih organizacija, trudili smo se da odaberemo verodostojne, ali precizna metodologija iza nekih od brojeva nije uvek javno dostupna.}

Što se tiče informatičkog \textbf{obrazovanja} mogao se uzeti primer našeg fakulteta, saznati o broju upisanih devojaka i mladića, desetak godina unazad i odnosa broja upisanih i završenih studija, po polnoj strukturi i to dati kao primer rodne ravnopravnosti. Isto tako je trebalo utvrditi polnu strukturu nastavnog kadra.

\odgovor{Sigurni smo da takav podatak ne bi škodio, ali s obzirom da pišemo o ravnopravnosti u informatici o svetu, ne mislimo da je ova tema dovoljno značajna i reprezentativna da zauzima ograničen prostor koji nam je dat.}

U podsekciji \textbf{razlozi za nejednaku zastupljenost} su brojni kao što se kaže u radu i naveden je primer Malezije gde je rodna ravnopravnost postignuta u informatici, pa je trebalo pokušati naći razlog za to.

\odgovor{Izbacili smo poslednju rečenicu iz ovog pasusa, a dodali ,,...i predlaže sličan pristup za druge države. Zaključci iz ovog istraživanja se teško mogu generalizovati, ali nas podsećaju da je drugačija realnost moguća kao rezultat pažljivo odabranih intervencija i politika.'', kako bi pojasnili nameru iza spominjanja ovog primera. Zainteresovanom čitaocu toplo preporučujemo citirani rad, ali ne smatramo da je ulaženje u specifičnosti malezijskog IT sektora opravdano trošenje prostora u ovom radu.}

U delu gde se govori o \textbf{razlikama između akademije i industrije} za Ameriku je dat primer i za akademiju  i za industriju a za Evropu samo za akadeniju.

\odgovor{Dodali smo rečenicu ,,S obzirom na ovoliki raspon, nije moguće generalizovati bilo koji konkretan zaključak na ceo kontinent, posebno pošto postoje države u kojima je situacija skoro ravnopravna.'' na kraj prvog pasusa u ovom odeljku, s obzirom da nismo našli bolje podatke. Takođe smo dodali rečenicu ,,Ovi razlozi u opštem slučaju ne važe za Evropu.'' na kraju drugog pasusa.}

Kako bi se poboljšala \textbf{rodna ravnopravnost u informatici u Srbiji} trebalo je dati predloge načina za njeno poboljšanje i načine da se to uradi.

\odgovor{Na kraj ovog odeljka (str. 7) smo dodali rečenicu: ,,U nastavku ćemo prikazati neke od načina na koji se ovaj problem rešava u Evropi i Americi i koje možemo primeniti i u Srbiji, ali kvalitetni i konkretni predlozi zahtevaju više istraživanja o tome šta bi moglo da funkcioniše na ovim prostorima.'' kojom pojašnjavamo zašto nismo posvetili više vremena ovoj temi i pravimo uvod za naredni odeljak.}

U \textbf{zaključku} je trebalo naglasiti da rodna ravnopravnost pretpostavlja da u jednom društvu, zajednici ili organizaciji treba da postoji jednaka mogućnost kako za žene, muškarce tako i za osobe drugačijih rodnih identiteta. Trebalo je naglasiti da programiranje treba shvatiti kao i ostale poslove do čijeg se rezultata dolazi timskim radom muškaraca i žena. Društva koja su najdalje napredovala po pitanju informatike su ona u kojima je sposobnost razumevanja novih tehnologija podjednako raspoređena na oba pola.

\odgovor{Dodali smo: ,,Rodna ravnopravnost podrazumeva da u jednom društvu, a samim tim i u svakoj zajednici i organizaciji, treba da postoji jednaka mogućnost kako za muškarce, tako i za žene i osobe drugih rodnih identiteta i smatra se osnovnim ljudskim pravom. Naučna istraživanja i inicijative su i dalje fokusirani isključivo na odnos muškaraca i žena i videli smo da se položaj žena...'' na početak prvog pasusa. Nismo sigurni šta je namera druge rečenice i čime je potkrepljena. Poslednja rečenica možda ima intuitivno smisla, ali nismo uspeli da nađemo empirijsku potvrdu za to tvrđenje; ono je posebno osetljivo u svetlu paradoksa jednakosti polova koji je pomenut na strani 6, a koji je i sam kontroverzan (ali deluje da ga rezultati jed(i)nog većeg internacionalnog istraživanja na ovu temu podržavaju: \emph{Tao et al, Are gender differences in vocational interests universal?: Moderating effects of cultural dimensions, Sex Roles, 2022, doi: 10.1007/s11199-022-01318-w}; nismo uspeli da nađemo punu verziju ovog rada u datom roku, te ga ne navodimo kao referencu u našem seminarskom).}

\section{Sitne primedbe}
% Напишете своја запажања на тему штампарских-стилских-језичких грешки

Mislim da bi više tabelarnog i grafičkog prikaza dalo bolji uvid u temu. 

\odgovor{Slažemo se, međutim oba tipa prikaza su prostorno zahtevni, a smatramo da je taj prostor bolje popuniti tekstom (iako je možda manje prijemčivo čitaocu).}

U tabeli 1 treba hronološki poređati redove po godini osnivanja preduzeća.

\odgovor{Ispravljeno.}

Citate je trebalo navoditi pre tačaka i ostalih interpunkcijskih znakova.

\odgovor{Ispravljeno.}

Trebalo je koristiti \verb|\usepackage{xurl}| (ili neko drugo rešenje) za pravilno prelamanje linkova u nove redove u literaturi.

\odgovor{Ispravljeno.}

\section{Provera sadržajnosti i forme seminarskog rada}
% Oдговорите на следећа питања --- уз сваки одговор дати и образложење

\begin{enumerate}
\item Da li rad dobro odgovara na zadatu temu?\\
\odgovor{Da, skoro u potpunosti odgovara zadatoj temi s obzirom na broj istraživanja i broja dostupnih podataka.}
\item Da li je nešto važno propušteno?\\
\odgovor{Da, opšta definicija rodne ravnopravnosti kao uvod u seminiraski rad.}
\item Da li ima suštinskih grešaka i propusta?\\
\odgovor{Ne, a s obzirom na mali broj istraživanja ipak se može sagledati realno stanje rodne neravnopravnosti.}
\item Da li je naslov rada dobro izabran?\\
\odgovor{Jeste, zato što je rad ukazao na razne nedostatke i probleme rodne neravnopravnosti u informatici u svetu.}
\item Da li sažetak sadrži prave podatke o radu?\\
\odgovor{Da, sažetak sadrži sve potrebne podatke koji su se diskutovale u radu.}
\item Da li je rad lak-težak za čitanje?\\
\odgovor{Da, rad je dobro napisan, dobro struktuiran, jasan i čitljiv.}
\item Da li je za razumevanje teksta potrebno predznanje i u kolikoj meri?\\
\odgovor{Potrebna je opšta informisanost o odnosu polova u svetu i Srbiji, odnos u društvu, obrazovanju, zaposlenosti po delatnostima, posebno u informatici.}
\item Da li je u radu navedena odgovarajuća literatura?\\
\odgovor{Da, navedena je sva korišćena literatura, uglavnom iz istraživanja sa raznih IT fakulteta i IT kompanija.}
\item Da li su u radu reference korektno navedene?\\
\odgovor{Da, navedeno je odakle potiču istraživanja i podaci.}
\item Da li je struktura rada adekvatna?\\
\odgovor{Da, ima sve što treba da ima jedan dobro strktuiran seminarski rad - naslovna strana, sadržaj, uvod, glavni deo po poglavljima, zaključak i literaturu.}
\item Da li rad sadrži sve elemente propisane uslovom seminarskog rada (slike, tabele, broj strana...)?\\
\odgovor{Da, odgovara zahtevima profesorke.}
\item Da li su slike i tabele funkcionalne i adekvatne?\\
\odgovor{Da, jesu funkcionalne. Na slici 1 procenat žena na rukovodećim pozicijama, po istraživanju britanskog časpisa, u MINT
disciplinama je veoma nizak. Iz tog podatka može se videti da je u pitanju rodna neravnoptavnost. U tabeli 1, dat je pregled nekoliko organizacija koje se bave rodnom ravnopranošću u informatici gde su dati odgovarajući predlozi i preporuke radi poboljšanja iste.}
\end{enumerate}

\section{Ocenite sebe}
% Napišite koliko ste upućeni u oblast koju recenzirate: 
% a) ekspert u datoj oblasti
% b) veoma upućeni u oblast
% c) srednje upućeni
% d) malo upućeni 
% e) skoro neupućeni
% f) potpuno neupućeni
% Obrazložite svoju odluku

Ja kao recenzent seminarskog rada, mojih kolega i koleginica sa fakulteta, o rodnoj ravnopravnosti u informatici u svetu, pokušao sam da dam korektno i objektivno svoje mišljenje. Trudio sam se da pažljivo čitam i da tokom čitanja pravim beleške i na osnovu njih napraviti skicu recenzije. Konačnu recenziju sam napisao tek kada sam seminarski rad pročitao više puta.

Iako se od recenzenata očekuje da bude kompetentan u oblasti teme koju recenzira, smatram da nemam dovoljno znanja i da sam nedovoljno upućen u temu seminarskog rada. S obzirom na to, nisam siguran koliko sam uspeo da procenim vrednost i tačnost prezentovanih podataka i rezultata iz navedenih istraživanja.

Cilj recenziranja je dobiti objektivnu procenu rada ali ne znam koliko sam uspeo u tome, jedino u šta sam siguran je to da u seminarskom radu nema lažnih, neverodostojnih ili falsifikovanih rezultata.


\chapter{Dodatne izmene}
\odgovor{
\begin{itemize}
\item Ispravljene reference: na nekoliko mesta je falio naziv časopisa, izdavača, bio je pogrešan tip reference ili je došlo do greške u kucanju u imenu nekog od polja (čime je ono bilo izostavljeno).
\item Ispravljen prelom u poslednjem pasusu strane 2: reč ,,računara'' u prvoj rečenici je izlazila iz margine, dodata prelomna tačka posle drugog sloga. Dodat \texttt{\textbackslash{}sloppy} i ručno prelamanje na prvom slogu reči ,,muškaraca'' u osmom pasusu u odeljku 2, kako red ne bi izlazio iz margine. Reč ,,prijavljeni'' zamenjena rečju ,,kandidati'' koju smatramo boljom, a pritom nam štedi prostor.
\item U odeljku 2, treći pasus, treća rečenica, zamenjena reč ,,školovaniji'' sa ,,školovani'': ovime značenje ostaje približno isto, a dobija se jedan red.
\item {U odeljku 2, pretposlednji pasus, reč ,,2000-ih'' je zamenjena reči ,,veka'', kako bismo uštedeli jedan red.}
\item {U odeljku 2, poslednji pasus, reči ,,oba pola'' zamenjene inkluzivnijom alternativom ,,polova'', koja pritom štedi prostor.}
\item {U odeljku 3.1 u drugom pasusu pojedine reči zamenjene kraćim sinonimima, kako bi pasus stao na jednu stranu.}
\item {U odeljku 3.1 na kraju prvog pasusa tekst u zagradi zamenjen sa ,,na svim stepenima studija''.}
\item {U odeljku 3.2 izbačen deo ,,,počev od obrazovanja'' na kraju prve rečenice; može biti konfuzan, nije neophodan, a njegovo izbacivanje nam štedi jedan red.}
\item {U odeljku 3.2, u prvoj rečenici trećeg pasusa, reč ,,zainteresovanost'' zamenjena sinonimom ,,interesovanje'' koji je adekvatniji i štedi nam jedan red.}
\item {U odeljku 3.2, u drugoj rečenici pretposlednjeg pasusa izbačena reč ,,zvanična'' u sintagmi ,,zvanična statistika'': ovime se štedi jedan red, a ne gubi na značenju.}
\item {U odeljku 3.2, u poslednjem pasusu, u prvoj rečenici zamenjen tekst ,,što bismo mogli intuitivno očekivati'' tekstom ,,što je intuitivno očekivano'' i poslednja rečenica prepravljena da glasi: ,,Ova rasprava nije dobila epilog i ne postoji širi naučni konsenzus o ovome'' čime se značenje ne menja a štedi se jedan red.}
\item U odeljku 3.3, u drugom pasusu, početak druge rečenice preformulisan da glasi: ,,Neke od razloga zašto je razlika u polovima ovde veća možemo potražiti...''
\item U odeljku 4, u prvom pasusu, treća, četvrta i peta rečenica su spojene u jednu i izbačen je središnji deo koji nije značajnije doprinosio značenju pasusa. Poslednja rečenica je zamenjena nešto kraćim ekvivalentom: ,,Ukoliko se osvrnemo na podatke iz naše okoline, vidimo da je sve veći broj studentkinja koje se prijavljuju za tehničke i prirodno-matematičke fakultete, ali su devojke i dalje u manjini.''
\item U odeljku 4, u trećem pasusu prva rečenica je pojednostavljena i sada glasi: ,,Pored ovoga, problematika koju primećujemo jeste što iako u Srbiji postoji sve više IT stručnjaka koji rade u stranim startapima, te rade od kuće uz pogodnosti visokih plata, jako je čest slučaj da ove strane kompanije nemaju sedište u Srbiji i ne pružaju mogućnost plaćenog porodiljskog odustva.''
\item U odeljku 5, u drugom pasusu izbačen veznik ,,kao'' iz prve rečenice. Reč ,,pobednicima'' zamenjena rečju ,,dobitnicima''. Poslednja rečenica zamenjena ekvivalentnom kraćom: ,,Stenford je načinio veliki napredak redizajnirajući uvodne kurseve i čineći ih pristupačnijim široj publici i sa 12.5\% žena u studentskoj populaciji 2008, došli su do 21\% 2013.''
\item U odeljku 5, u četvrtom pasusu izbačena prva rečenica, a druga preformulisana i sada glasi: ,,Ovakve inicijative nisu ograničene samo na koledže: tabelarni prikaz nekih organizacija koje su se bavile ovim pitanjem je dostupan u tabeli 1''. Ovime se rad skraćuje i celokupan sadržaj staje na 12 strana.
\end{itemize}
}
%Ovde navedite ukoliko ima izmena koje ste uradili a koje vam recenzenti nisu tražili. 

\end{document}
